%-------------------------
% Resume in Latex
% Author : Aras Gungore
% License : MIT
%------------------------

\documentclass[letterpaper,11pt]{article}

\usepackage{latexsym}
\usepackage[empty]{fullpage}
\usepackage{titlesec}
\usepackage{marvosym}
\usepackage[usenames,dvipsnames]{color}
\usepackage{verbatim}
\usepackage{enumitem}
\usepackage[hidelinks]{hyperref}
\usepackage{fancyhdr}
\usepackage[english]{babel}
\usepackage{tabularx}
\usepackage{hyphenat}
\usepackage{fontawesome}
\input{glyphtounicode}


%---------- FONT OPTIONS ----------
% sans-serif
% \usepackage[sfdefault]{FiraSans}
% \usepackage[sfdefault]{roboto}
% \usepackage[sfdefault]{noto-sans}
% \usepackage[default]{sourcesanspro}

% serif
% \usepackage{CormorantGaramond}
% \usepackage{charter}


\pagestyle{fancy}
\fancyhf{} % clear all header and footer fields
\fancyfoot{}
\renewcommand{\headrulewidth}{0pt}
\renewcommand{\footrulewidth}{0pt}

% Adjust margins
\addtolength{\oddsidemargin}{-0.5in}
\addtolength{\evensidemargin}{-0.5in}
\addtolength{\textwidth}{1in}
\addtolength{\topmargin}{-.5in}
\addtolength{\textheight}{1.0in}

\urlstyle{same}

\raggedbottom
\raggedright
\setlength{\tabcolsep}{0in}

% Sections formatting
\titleformat{\section}{
  \vspace{-4pt}\scshape\raggedright\large
}{}{0em}{}[\color{black}\titlerule \vspace{-5pt}]

% Ensure that generate pdf is machine readable/ATS parsable
\pdfgentounicode=1

%-------------------------
% Custom commands

\newcommand{\resumeItem}[1]{
  \item\small{
    {#1 \vspace{-2pt}}
  }
}


\newcommand{\resumeSubheading}[4]{
  \vspace{-2pt}\item
    \begin{tabular*}{0.97\textwidth}[t]{l@{\extracolsep{\fill}}r}
      \textbf{#1} & #2 \\
      \textit{\small#3} & \textit{\small #4} \\
    \end{tabular*}\vspace{-7pt}
}


\newcommand{\resumeSubSubheading}[2]{
    \vspace{-2pt}\item
    \begin{tabular*}{0.97\textwidth}{l@{\extracolsep{\fill}}r}
      \textit{\small#1} & \textit{\small #2} \\
    \end{tabular*}\vspace{-7pt}
}


\newcommand{\resumeEducationHeading}[6]{
  \vspace{-2pt}\item
    \begin{tabular*}{0.97\textwidth}[t]{l@{\extracolsep{\fill}}r}
      \textbf{#1} & #2 \\
      \textit{\small#3} & \textit{\small #4} \\
      \textit{\small#5} & \textit{\small #6} \\
    \end{tabular*}\vspace{-5pt}
}


\newcommand{\resumeProjectHeading}[2]{
    \vspace{-2pt}\item
    \begin{tabular*}{0.97\textwidth}{l@{\extracolsep{\fill}}r}
      \small#1 & #2 \\
    \end{tabular*}\vspace{-7pt}
}


\newcommand{\resumeOrganizationHeading}[4]{
  \vspace{-2pt}\item
    \begin{tabular*}{0.97\textwidth}[t]{l@{\extracolsep{\fill}}r}
      \textbf{#1} & \textit{\small #2} \\
      \textit{\small#3}
    \end{tabular*}\vspace{-7pt}
}

\newcommand{\resumeSubItem}[1]{\resumeItem{#1}\vspace{-4pt}}

\renewcommand\labelitemii{$\vcenter{\hbox{\tiny$\bullet$}}$}

\newcommand{\resumeSubHeadingListStart}{\begin{itemize}[leftmargin=0.15in, label={}]}
\newcommand{\resumeSubHeadingListEnd}{\end{itemize}}
\newcommand{\resumeItemListStart}{\begin{itemize}}
\newcommand{\resumeItemListEnd}{\end{itemize}\vspace{-5pt}}

%-------------------------------------------
%%%%%%  RESUME STARTS HERE  %%%%%%%%%%%%%%%%%%%%%%%%%%%%


\begin{document}

%---------- HEADING ----------

\begin{center}
    \textbf{\Huge \scshape Ray Yan} \\ \vspace{1pt}
    \small \textit{(Legal name: Kin Long Yan)} \\ \vspace{3pt}
    \small
    \faAt \hspace{.5pt} \href{mailto:ray.kl.yan0409@gmail.com}{ray.kl.yan0409@gmail.com}
    $|$
    \faGithub \hspace{.5pt} \href{https://github.com/chkpcc1a035}{GitHub}
    $|$
    \faLinkedinSquare \hspace{.5pt} \href{https://www.linkedin.com/in/kin-long-yan-7a6950168/}{LinkedIn}
\end{center}



%----------- EDUCATION -----------

\section{Education}
  \vspace{3pt}
  \resumeSubHeadingListStart
    
    \resumeSubheading
      {UOW College Hong Kong}{Hong Kong}
      {Associate of Science in Creative and Interactive Media Production}{2019 \textbf{--} 2021}
    
  \resumeSubHeadingListEnd



%----------- SKILLS -----------

\section{Skills}
  \vspace{2pt}
  \resumeSubHeadingListStart
    \small{\item{
        
        \textbf{Programming Languages:}{ TypeScript, Python, Java (J2EE, Java 7, 8, 15), C++, Shell, Rust} \\ \vspace{3pt}
        
        \textbf{Web Backend \& Databases:}{ Node.js, Next.js, Express, Fastify, Vue, Remix, MongoDB, MySQL, PostgreSQL, Redis, Django} \\ \vspace{3pt}
        
        \textbf{Web Frontend:}{ React, Vue, NextJS, RemixJS} \\ \vspace{3pt}
        
        \textbf{Automation \& Testing:}{ Selenium, Playwright} \\ \vspace{3pt}
        
        \textbf{Monitoring \& Infrastructure:}{ Grafana, Active Directory, JIRA} \\ \vspace{3pt}
        
        \textbf{Cloud:}{ AWS, GCP, Docker, Kubernetes} \\ \vspace{3pt}
        
        \textbf{Languages:}{ Chinese (Cantonese): Native, Chinese (Mandarin): Native, English: Fluent} \\ \vspace{3pt}
        
    }}
  \resumeSubHeadingListEnd



%----------- EXPERIENCE -----------

\section{Experience}
  \vspace{3pt}
  \resumeSubHeadingListStart

    \resumeSubheading
      {PCCW Solutions}{Hong Kong}
      {DevOps Engineer}{Mar 2025 \textbf{--} Current}
        \resumeItemListStart
            \resumeItem{Site Reliability Engineer for largest Hong Kong eMPF Platform - a central and integrated electronic platform to standardize, streamline and automate MPF scheme administration work.}
            \resumeItem{Design, implement, and maintain CI/CD pipelines using Jenkins and Bitbucket for deploying Web Applications, Android, and iOS applications in enterprise environments. Developed complex Jenkins Groovy scripts for pipeline automation, custom build steps, and deployment orchestration. Performed pipeline reconfiguration, ongoing maintenance, future pipeline development, and troubleshooting of deployment issues. Created automation scripts using Shell, Python, and extensive Jenkins Groovy scripting for CI/CD processes while implementing containerization solutions.}
            \resumeItem{Manage software deployment lifecycle on Red Hat OpenShift with Oracle Database environments, ensuring secure and reliable delivery processes.}
            \resumeItem{Integrate security tools including CyberArk, HashiCorp Vault, and Splunk to safeguard sensitive credentials and data throughout CI/CD processes.}
            \resumeItem{Collaborate with Application Teams and Testing Teams to ensure seamless version control, automated testing, and environment setup for enterprise applications.}
            \resumeItem{Deploy and configure applications on Nginx, MuleSoft API Gateway, Red Hat Directory Server, RHSSO, and Solace within private cloud infrastructure.}
            \resumeItem{Monitor CI/CD pipeline performance using Grafana dashboards and troubleshoot issues during build, testing, and deployment phases to maintain high availability.}
            \resumeItem{Manage user authentication and authorization through Active Directory integration and coordinate project workflows using JIRA for task tracking and sprint management.}
        \resumeItemListEnd

    \resumeSubheading
      {Ntuple Global Limited}{Hong Kong}
      {Full-stack Developer/Senior Software Developer (Remote from Canada)}{Oct 2022 \textbf{--} Mar 2025}
        \resumeItemListStart
            \resumeItem{Designed and implemented system architecture to extract terabytes of data for ETL system development for the largest public railway transport in Hong Kong.}
            \resumeItem{Developed various integrations for different types of data warehouse and services.}
            \resumeItem{Developed API server using Python and Selenium/Playwright for web scraping and automation.}
            \resumeItem{Managed data storage in MySQL and Amazon S3 for efficient and secure handling.}
            \resumeItem{Conducted comprehensive training and provided ongoing support for non-technical staff.}
            \resumeItem{Provided comprehensive full-stack web development services for company clients, bringing sustainable income from multiple F\&B and CRM projects.}
            \resumeItem{Managed AWS EKS, EC2, bringing intermediate level cloud infrastructure management to the company. Configured load balancers and implemented bandwidth control within EKS clusters. Utilized Infrastructure as Code (IaC) with Terraform and Ansible for automated infrastructure provisioning. Designed CI/CD pipelines using GitHub Actions with AWS Systems Manager (SSM) integration.}
            \resumeItem{Developed online ordering systems with modern security standards and payment gateway integration with Mastercard, Stripe, and GlobalPayment.}
            \resumeItem{Successfully submitted and managed tender proposals for public sector projects, demonstrating expertise in government procurement processes and compliance requirements.}
            \resumeItem{Led and mentored development teams of 2-3 people, coordinating project deliverables and ensuring code quality standards across multiple client projects.}
        \resumeItemListEnd

    \resumeSubheading
      {PCCW Solutions Limited}{Hong Kong}
      {Software Engineer}{Oct 2021 \textbf{--} Oct 2022}
        \resumeItemListStart
            \resumeItem{Provided on-site production and User Acceptance Testing (UAT) support for Oracle Weblogic Server to maintain server stability and reliability.}
            \resumeItem{Maintained current Java codebase and fulfilled client requests.}
            \resumeItem{Developed a backend program with Java to generate CSV files from the database for security analysis.}
        \resumeItemListEnd

    \resumeSubheading
      {Chun Wo Limited}{Hong Kong}
      {Web Developer}{Jul 2021 \textbf{--} Oct 2021}
        \resumeItemListStart
            \resumeItem{Designed and created a full-stack web application with PHP framework Laravel, along with React.js and Tailwind as the frontend.}
            \resumeItem{Performed basic network security testing to find vulnerabilities and patch before production.}
        \resumeItemListEnd
    
  \resumeSubHeadingListEnd



%----------- PERSONAL WORK -----------

\section{Personal Work}
  \vspace{2pt}
  \resumeSubHeadingListStart
    \small{\item{
        \textbf{Freelance Vendor - Hong Kong TVP Programme:}{ Developed full-stack web application using Next.js with modern frontend technologies including Tailwind CSS and ShadCN UI components} \\ \vspace{3pt}
        
        \textbf{Mobile Application Development:}{ Built iOS applications using Xcode with paid Apple Developer account and developed Android applications using Android Studio} \\ \vspace{3pt}
        
        \textbf{Operating Systems Expertise:}{ Extensive macOS user with multiple years of Unix-like macOS experience and comprehensive Linux expertise across various distributions including Arch Linux and Kali Linux} \\ \vspace{3pt}
        
        \textbf{Machine Learning \& AI:}{ RAG development with GraphRAG and MCP, n8n workflow automation experience, small-scale vLLM deployment, and CUDA tools proficiency including PyTorch SDK} \\ \vspace{3pt}
        
    }}
  \resumeSubHeadingListEnd



%----------- AWARDS & ACHIEVEMENTS -----------

\section{Certifications}
  \vspace{2pt}
  \resumeSubHeadingListStart
    \small{\item{
        \textbf{Google IT Support:}{ Professional Certificate/Specialization - EKHGD87ZDBYX} \\ \vspace{3pt}

        \textbf{AWS DevOps on AWS:}{ Professional Certificate/Specialization - 589 GROWN NW9} \\ \vspace{3pt}
        
        \textbf{AWS Cloud Solutions Architect:}{ Professional Certificate/Specialization - Z5Y82HAUG9JV} \\ \vspace{3pt}
        
        \textbf{AWS Certified Cloud Practitioner:}{ CLP - Expiration: 2027-07-14 - ID: 2f23a7d4bf484e94b5a893fc29f24ae6} \\ \vspace{3pt}
        
        \textbf{AWS Certified Solutions Architect – Associate:}{ SAA - Expiration: 2027-07-20 - ID: 6571bd4dcd3542c084961a64105304f1} \\ \vspace{3pt}
        
        \textbf{Linux Foundation Certified Kubernetes Application Developer:}{ CKAD - ID: LF-sxumuxu0ty}
    }}
  \resumeSubHeadingListEnd



%----------- RELEVANT COURSEWORK -----------

% \section{Relevant Coursework}
  % \vspace{2pt}
  % \resumeSubHeadingListStart
    % \small{\item{
        % \textbf{Major coursework:}{ Materials Science, Electrical Circuits I-II, Digital System Design, Numerical Methods, Probability Theory, Electronics I-II, Signals and Systems, Electromagnetic Field Theory, Energy Conversion, System Dynamics and Control, Communication Engineering, Pattern Recognition, Introduction to Digital Signal Processing, Introduction to Digital Communications, Introduction to Database Systems, Introduction to Image Processing, Machine Vision} \\ \vspace{3pt}
        
        % \textbf{Minor coursework:}{ Discrete Computational Structures, Introduction to Object-Oriented Programming, Data Structures and Algorithms, Computer Organization, Fundamentals of Software Engineering}
    % }}
  % \resumeSubHeadingListEnd



%----------- CERTIFICATES -----------

% \section{Certificates}
  % \resumeSubHeadingListStart
    
    % \resumeOrganizationHeading
      % {Procter \& Gamble VIA Certificate Program}{Feb 2022}{Project Management and Personal Productivity}
    
  % \resumeSubHeadingListEnd



%----------- ORGANIZATIONS -----------

% \section{Organizations}
  % \resumeSubHeadingListStart
    
    % \resumeOrganizationHeading
      % {Institute of Electrical and Electronics Engineers (IEEE)}{Feb 2022 -- Present}{Student Member}
    
  % \resumeSubHeadingListEnd



%----------- HOBBIES -----------

% \section{Hobbies}
  % \resumeSubHeadingListStart
    % \small{\item{Basketball, Swimming, Fitness, Eight-ball, Horology}}
  % \resumeSubHeadingListEnd



%----------- REFERENCES -----------

% \section{References}
  % \vspace{2pt}
  % \resumeSubHeadingListStart
    % \item{References available upon request.}
  % \resumeSubHeadingListEnd



%-------------------------------------------
\end{document}
